%%%%%%%%%%%%%%%%%%%%%%%%%%%%%%%%%%%%%%%%%
% Developer CV
% LaTeX Template
% Version 1.0 (28/1/19)
%
% This template originates from:
% http://www.LaTeXTemplates.com
%
% Authors:
% Jan Vorisek (jan@vorisek.me)
% Based on a template by Jan Küster (info@jankuester.com)
% Modified for LaTeX Templates by Vel (vel@LaTeXTemplates.com)
%
% License:
% The MIT License (see included LICENSE file)
%
%%%%%%%%%%%%%%%%%%%%%%%%%%%%%%%%%%%%%%%%%

%----------------------------------------------------------------------------------------
%	PACKAGES AND OTHER DOCUMENT CONFIGURATIONS
%----------------------------------------------------------------------------------------

\documentclass[9pt]{developercv} % Default font size, values from 8-12pt are recommended

%----------------------------------------------------------------------------------------

\begin{document}

%----------------------------------------------------------------------------------------
%	TITLE AND CONTACT INFORMATION
%----------------------------------------------------------------------------------------

\begin{minipage}[t]{0.45\textwidth} % 45% of the page width for name
	\vspace{-\baselineskip} % Required for vertically aligning minipages

	% If your name is very short, use just one of the lines below
	% If your name is very long, reduce the font size or make the minipage wider and reduce the others proportionately
	\colorbox{black}{{\HUGE\textcolor{white}{\textbf{\MakeUppercase{Felix}}}}} % First name

	\colorbox{black}{{\HUGE\textcolor{white}{\textbf{\MakeUppercase{Hoffmann}}}}} % Last name

	\vspace{6pt}

	{\huge Dualer Student} % Career or current job title
\end{minipage}
\begin{minipage}[t]{0.275\textwidth} % 27.5% of the page width for the first row of icons
	\vspace{-\baselineskip} % Required for vertically aligning minipages

	% The first parameter is the FontAwesome icon name, the second is the box size and the third is the text
	% Other icons can be found by referring to fontawesome.pdf (supplied with the template) and using the word after \fa in the command for the icon you want
	\icon{Asterisk}{12}{Berlin}\\
	\icon{MapMarker}{12}{Stuttgart}\\
	\icon{Phone}{12}{+49 176 2147 0118}\\
\end{minipage}
\begin{minipage}[t]{0.275\textwidth} % 27.5% of the page width for the second row of icons
	\vspace{-\baselineskip} % Required for vertically aligning minipages

	% The first parameter is the FontAwesome icon name, the second is the box size and the third is the text
	% Other icons can be found by referring to fontawesome.pdf (supplied with the template) and using the word after \fa in the command for the icon you want
  \icon{At}{12}{\href{mailto:felix.hoffmann@hpe.com}{felix.hoffmann@hpe.com}}\\
	\icon{Globe}{12}{\href{https://felixemmanuel.de}{felixemmanuel.de}}\\
	\icon{Github}{12}{\href{https://github.com/felixhoffmnn}{felixhoffmnn}}\\
\end{minipage}

\vspace{0.5cm}

%----------------------------------------------------------------------------------------
%	INTRODUCTION, SKILLS AND TECHNOLOGIES
%----------------------------------------------------------------------------------------

\cvsect{Wer bin ich?}

\begin{minipage}[t]{0.4\textwidth} % 40% of the page width for the introduction text
	\vspace{-\baselineskip} % Required for vertically aligning minipages

	\lorem \lorem \lorem \lorem \lorem\\ % Dummy text
\end{minipage}
\hfill % Whitespace between
\begin{minipage}[t]{0.5\textwidth} % 50% of the page for the skills bar chart
	\vspace{-\baselineskip} % Required for vertically aligning minipages
	\begin{barchart}{5.5}
		\baritem{Python}{70}
		\baritem{TypeScript}{65}
		\baritem{Git}{50}
		\baritem{LaTeX}{40}
		\baritem{MS Office}{55}
	\end{barchart}
\end{minipage}

% \begin{center}
% 	\bubbles{5/Eclipse, 6/git, 4/Office, 3/Inkscape, 3/Blender}
% \end{center}

%----------------------------------------------------------------------------------------
%	EXPERIENCE
%----------------------------------------------------------------------------------------

\cvsect{Berufserfahrung}

\begin{entrylist}
	\entry
	{1/2023 -- 2/2023}
	{Front-end Development}
	{Poinnext Services (HPE)}
	{\lorem\\ \texttt{React}\slashsep\texttt{Typescript}\slashsep\texttt{Optimierung von Ladezeiten}}
	\entry
	{6/2022 -- 9/2022}
	{GPU Benchmarking}
	{Hewlett Packard Labs (HPE)}
	{\lorem\\ \texttt{Determined AI}\slashsep\texttt{Kubernetes}\slashsep\texttt{Machine Learning}\slashsep\texttt{Python}}
	\entry
	{12/2021 -- 3/2022}
	{Front-end Development}
	{Digital Edge Ratingen (DXC)}
	{\lorem\\ \texttt{React}\slashsep\texttt{Three.js}\slashsep\texttt{3D-Visualisierung}}
	\entry
	{3/2021 -- 5/2021}
	{No-Code/Low-Code Development}
	{Digital Service Innovation (DXC)}
	{\lorem\\ \texttt{Power Apps}\slashsep\texttt{Knowledge Management}}
	\entry
	{12/2021 -- 3/2022}
	{Analyse von Monitoring-Tools}
	{Pointnext Services (HPE)}
	{\lorem\\ \texttt{Monitoring}}
\end{entrylist}

\vspace{0.5cm}

\begin{entrylist}
	\entry
	{2017, 2016, 2015}
	{Fertigung, Qualitätskontrolle, Einkauf}
	{ASM}
	{\lorem}
	\entry
	{2017}
	{Flugzeugfertigung}
	{Stemme AG}
	{\lorem}
	\entry
	{2016}
	{Hardwareentwicklung}
	{alpha-board}
	{\lorem}
\end{entrylist}

%----------------------------------------------------------------------------------------
%	EDUCATION
%----------------------------------------------------------------------------------------

\cvsect{Schulische Ausbildung}

\begin{entrylist}
	\entry
	{2020 -- 2023}
	{Bachelor of Science}
	{DHBW Stuttgart}
	{Duales Studium im Bereich Informatik\\ \texttt{Kurssprecher}\slashsep\texttt{Mitglied in der JAV}}
	\entry
	{2017 -- 2020}
	{Abitur}
	{Oberstufenzentrum Märkisch Oderland}
	{Abitur in 3 Jahren mit technischem Schwerpunkt (Maschinen- und Kommunikationstechnik)\\ \texttt{Klassensprecher}\slashsep\texttt{Mitglied im Schülerrat}}
	\entry
	{2006 -- 2017}
	{Mittlerer Schulabschluss}
	{BundStift Schule}
	{Anerkannte Ersatzschule mit kreativem Schwerpunkt}
\end{entrylist}

%----------------------------------------------------------------------------------------
%	International experience
%----------------------------------------------------------------------------------------

\cvsect{Auslandserfahrung}

\begin{entrylist}
	\entry
	{2022}
	{Hewlett Packard Labs}
	{San Francisco, USA}
	{Duales Studium im Bereich Informatik}
	\entry
	{2017}
	{Bildungs- und Begegnungsreise mit Workcamp}
	{Indien}
	{Reise für interkulturelle Zusammenführung und Hausbau-Workcamp als Hilfe für einen indigenen Stamm in Indien}
\end{entrylist}

%----------------------------------------------------------------------------------------
%	ADDITIONAL INFORMATION
%----------------------------------------------------------------------------------------

\begin{minipage}[t]{0.3\textwidth}
	\vspace{-\baselineskip} % Required for vertically aligning minipages

	\cvsect{Sprachen}

	\textbf{Deutsch} - Muttersprache\\
	\textbf{Englisch} - Verhandlungsicher\\
  \textbf{Französisch} - Grundlagen\\
  \textbf{Spansich} - Grundlagen
\end{minipage}
\hfill
\begin{minipage}[t]{0.3\textwidth}
	\vspace{-\baselineskip} % Required for vertically aligning minipages

	\cvsect{Interessen}

	Volleyball, Fotografie
\end{minipage}
\hfill
\begin{minipage}[t]{0.3\textwidth}
	\vspace{-\baselineskip} % Required for vertically aligning minipages

	\cvsect{Sonstiges}

	\lorem
\end{minipage}

%----------------------------------------------------------------------------------------

\end{document}
