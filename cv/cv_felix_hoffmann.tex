%%%%%%%%%%%%%%%%%%%%%%%%%%%%%%%%%%%%%%%%%
% Developer CV
% LaTeX Template
% Version 1.0 (28/1/19)
%
% This template originates from:
% http://www.LaTeXTemplates.com
%
% Authors:
% Jan Vorisek (jan@vorisek.me)
% Based on a template by Jan Küster (info@jankuester.com)
% Modified for LaTeX Templates by Vel (vel@LaTeXTemplates.com)
%
% License:
% The MIT License (see included LICENSE file)
%
%%%%%%%%%%%%%%%%%%%%%%%%%%%%%%%%%%%%%%%%%

%----------------------------------------------------------------------------------------
%	PACKAGES AND OTHER DOCUMENT CONFIGURATIONS
%----------------------------------------------------------------------------------------

\documentclass[9pt]{developercv} % Default font size, values from 8-12pt are recommended

%----------------------------------------------------------------------------------------

\begin{document}

%----------------------------------------------------------------------------------------
%	TITLE AND CONTACT INFORMATION
%----------------------------------------------------------------------------------------

\begin{minipage}[t]{0.45\textwidth} % 45% of the page width for name
	\vspace{-\baselineskip} % Required for vertically aligning minipages

	% If your name is very short, use just one of the lines below
	% If your name is very long, reduce the font size or make the minipage wider and reduce the others proportionately
	\colorbox{black}{{\HUGE\textcolor{white}{\textbf{\MakeUppercase{Felix}}}}} % First name

	\colorbox{black}{{\HUGE\textcolor{white}{\textbf{\MakeUppercase{Hoffmann}}}}} % Last name

	\vspace{6pt}

	{\huge Dualer Student} % Career or current job title
\end{minipage}
\begin{minipage}[t]{0.275\textwidth} % 27.5% of the page width for the first row of icons
	\vspace{-\baselineskip} % Required for vertically aligning minipages

	% The first parameter is the FontAwesome icon name, the second is the box size and the third is the text
	% Other icons can be found by referring to fontawesome.pdf (supplied with the template) and using the word after \fa in the command for the icon you want
	\icon{Asterisk}{12}{Berlin}\\
	\icon{MapMarker}{12}{Stuttgart}\\
	\icon{Phone}{12}{+49 176 2147 0118}\\
\end{minipage}
\begin{minipage}[t]{0.275\textwidth} % 27.5% of the page width for the second row of icons
	\vspace{-\baselineskip} % Required for vertically aligning minipages

	% The first parameter is the FontAwesome icon name, the second is the box size and the third is the text
	% Other icons can be found by referring to fontawesome.pdf (supplied with the template) and using the word after \fa in the command for the icon you want
  \icon{At}{12}{\href{mailto:felix.hoffmann@hpe.com}{felix.hoffmann@hpe.com}}\\
	\icon{Globe}{12}{\href{https://felixemmanuel.de}{felixemmanuel.de}}\\
	\icon{Github}{12}{\href{https://github.com/felixhoffmnn}{felixhoffmnn}}\\
\end{minipage}

\vspace{0.5cm}

%----------------------------------------------------------------------------------------
%	INTRODUCTION, SKILLS AND TECHNOLOGIES
%----------------------------------------------------------------------------------------

\cvsect{Ziele}

\begin{minipage}[t]{0.4\textwidth} % 40% of the page width for the introduction text
	\vspace{-\baselineskip} % Required for vertically aligning minipages

Mein Ziel ist es in Anschluss an meinen Informatik Bachelor einen Master zu absolvieren. Als Vertiefung möchte ich mich auf Data Engineering oder Software Engineering konzentrieren. \\
\end{minipage}
\hfill % Whitespace between
\begin{minipage}[t]{0.5\textwidth} % 50% of the page for the skills bar chart
	\vspace{-\baselineskip} % Required for vertically aligning minipages
	\begin{barchart}{5.5}
		\baritem{Python}{100}
		\baritem{TypeScript}{80}
		\baritem{Git}{70}
		\baritem{SQL}{65}
	\end{barchart}
\end{minipage}

% \begin{center}
% 	\bubbles{5/Eclipse, 6/git, 4/Office, 3/Inkscape, 3/Blender}
% \end{center}

%----------------------------------------------------------------------------------------
%	EXPERIENCE
%----------------------------------------------------------------------------------------

\cvsect{Berufserfahrung}

\begin{entrylist}
	\entry
	{1/2023 -- 2/2023}
	{Front-end Development}
	{Poinnext Services (HPE)}
	{Ladezeiten-Optimierung eines React-Frontends mittels Server-Side Pagination und Backend-Filtering.\\ \texttt{React}\slashsep\texttt{Typescript}\slashsep\texttt{Optimierung von Ladezeiten}}
	\entry
	{6/2022 -- 9/2022}
	{GPU Benchmarking}
	{Hewlett Packard Labs (HPE)}
	{Benchmarking von Algorithmen (z. B. Parameter Server und Ring All-Reduce) zum Trainieren von Machine Learning Modellen auf verteilten Systemen. Das Trainieren erfolgte auf der Maschine Learning Plattform „Determined AI“ mittels Kubernetes und Docker Containern. \\ \texttt{Determined AI}\slashsep\texttt{Kubernetes}\slashsep\texttt{Machine Learning}\slashsep\texttt{Python}}
	\entry
	{12/2021 -- 3/2022}
	{Front-end Development}
	{Digital Edge Ratingen (DXC)}
	{Umsetzung eines React Frontends zur 3D-Visualisierung von Diagrammen und Prozessen. State-managment erfolgte mittels Zustand und 3D Interaktion mittels React Three Fiber. \\ \texttt{React}\slashsep\texttt{Three.js}\slashsep\texttt{3D-Visualisierung}}
	\entry
	{3/2021 -- 5/2021}
	{No-Code/Low-Code Development}
	{Digital Service Innovation (DXC)}
	{Umsetzung einer zentralen Speicherlösung mittels einer Microsoft PowerApp. Interaktion anhand REST APIs mit Java Spring Backend. \\ \texttt{Power Apps}\slashsep\texttt{Knowledge Management}}
	\entry
	{10/2020 -- 11/2020}
	{Analyse von Monitoring-Tools}
	{Pointnext Services (HPE)}
	{Analyse von Open Source Monitoring Tools wie zum Beispiel Nagios und Zabbix. Die Erkenntnisse wurden in einer Tabelle und Projektarbeit dokumentiert. \\ \texttt{Monitoring}}
\end{entrylist}

\vspace{0.5cm}

\begin{entrylist}
	\entry
	{Seit 2021}
	{Selbständigkeit als Webentwickler}
	{Felix Emmanuel}
	{Umsetzung von statischen Webseiten, Fotografie und technische Beratung. \\ \texttt{Wordpress}\slashsep\texttt{Adobe XD}\slashsep\texttt{Fotografie}}
\end{entrylist}

%----------------------------------------------------------------------------------------
%	EDUCATION
%----------------------------------------------------------------------------------------

\cvsect{Ausbildung}

\begin{entrylist}
	\entry
	{2020 -- 2023}
	{Bachelor of Science}
	{DHBW Stuttgart}
	{Duales Studium im Bereich Informatik\\ \texttt{Kurssprecher}\slashsep\texttt{Mitglied in der JAV}}
	\entry
	{2017 -- 2020}
	{Abitur}
	{Oberstufenzentrum Märkisch Oderland}
	{Abitur in 3 Jahren mit technischem Schwerpunkt (Maschinen- und Kommunikationstechnik).\\ \texttt{Gesamtnote: 1,6}\slashsep\texttt{Klassensprecher}\slashsep\texttt{Mitglied im Schülerrat}}
	\entry
	{2006 -- 2017}
	{Mittlerer Schulabschluss}
	{BundStift Schulen}
	{Gymnasium mit kreativem Schwerpunkt}
\end{entrylist}

%----------------------------------------------------------------------------------------
%	International experience
%----------------------------------------------------------------------------------------

\newpage

\cvsect{Auslandserfahrung}

\begin{entrylist}
	\entry
	{2022}
	{Hewlett Packard Labs}
	{San Francisco, USA}
	{Auslandseinsatz im Network \& Distributed Systems Lab}
	\entry
	{2018}
	{Bildungs- und Begegnungsreise mit Workcamp}
	{Indien}
	{Reise für interkulturelle Zusammenführung und Hausbau-Workcamp als Hilfe für einen indigenen Stamm in Indien}
\end{entrylist}

%----------------------------------------------------------------------------------------
%	Projects
%----------------------------------------------------------------------------------------

\cvsect{Projekte}

\begin{entrylist}
  \entry
	{2023}
	{(Bachelorarbeit)}
	{DHBW Stuttgart \& Poinnext Services (HPE)}
	{Ladezeiten-Optimierung eines React-Frontends mittels adaptiven Backend-Filterring und gecachten Datensätzen.\\ \texttt{React}\slashsep\texttt{Typescript}\slashsep\texttt{Ladezeiten Optimierung}}
	\entry
	{2022 --  2023}
	{Studienarbeit}
	{DHBW Stuttgart}
	{NLP Projekt zur Erkennung von Parteien anhand von Tweets, Politischen Reden und Wahlprogrammen. Das trainierte Modell kann unter anderem zur Klassifikation Zeitungsartikeln eingesetzt werden.\\ \texttt{NLP}\slashsep\texttt{Python}\slashsep\texttt{Machine Learning}}
\end{entrylist}

%----------------------------------------------------------------------------------------
%	ADDITIONAL INFORMATION
%----------------------------------------------------------------------------------------

\begin{minipage}[t]{0.45\textwidth}
	\vspace{-\baselineskip} % Required for vertically aligning minipages

	\cvsect{Sprachen}

	\textbf{Deutsch} - Muttersprache\\
	\textbf{Englisch} - Verhandlungsicher\\
  \textbf{Französisch} - Grundlagen\\
  \textbf{Spanisch} - Grundlagen
\end{minipage}
\hfill
\begin{minipage}[t]{0.45\textwidth}
	\vspace{-\baselineskip} % Required for vertically aligning minipages

	\cvsect{Interessen}

	Volleyball, Fotografie und Web Design
\end{minipage}
% \hfill
% \begin{minipage}[t]{0.3\textwidth}
% 	\vspace{-\baselineskip} % Required for vertically aligning minipages

% 	\cvsect{Sonstiges}

% 	\lorem
% \end{minipage}

%----------------------------------------------------------------------------------------

\end{document}
